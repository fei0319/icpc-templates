\subsection{动态树}

\subsubsection{用法}

通过 \lstinline{LinkCutTree<Mono, TREE_N>} 实例化动态树类。

\begin{itemize}
\item \lstinline{Mono}:一个幺半群或半群类。需重载 \lstinline{*} 运算符。
\item \lstinline{TREE_N}:一个整数,值应\textbf{大于}需要使用的最大下标。
\end{itemize}

以下公共方法均允许不实现。

\begin{itemize}
\item \lstinline{link}:添加一条边。
\item \lstinline{cut}:切割一条边。
\item \lstinline{set}:设置一个节点的值。
\item \lstinline{query}:查询两点间路径上所有节点元素按顺序相乘的结果。
\item \lstinline{find}:查询一个节点所在树的根,可用于判断连通性。
\end{itemize}

\subsubsection{注意}

下标从 0 开始,但不支持使用 0 下标。

\subsubsection{性能}

$n, q = 2\times 10^5$ 约花费 1.49 个标准时间。

\subsubsection{代码}

\lstinputlisting{src/link_cut_tree.cpp}
