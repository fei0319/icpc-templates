\subsection{多项式}

高效实现的多项式类,提供一切多项式支持。

\subsubsection{用法}

通过一个 \lstinline{ModInt} 的实例类来实例化 \lstinline{Poly}。对应的模数需为适合的 NTT 模数,例如:

\begin{itemize}
\item $167772161 = 5 \times 2^{25} + 1$
\item $469762049 = 7 \times 2^{26} + 1$
\item $998244353 = 119 \times 2^{23} + 1$
\item $1004535809 = 479 \times 2^{21} + 1$
\end{itemize}

该实现需要特别的原根,可以通过 \lstinline{primitive_root.cpp} 计算。对于 $998244353$ 可使用 $31$,对于 $1004535809$ 可使用 $702606812$。使用时,用适合的原根替换第 17 行的 \lstinline{31}。

\lstinline{Poly} 继承了 \lstinline{std::vector},因此支持一切 \lstinline{std::vector} 的功能。可能需要在 \lstinline{Poly} 的 public 区添加 \lstinline{using std::vector<Z>::method} 来导入相应方法。

该实现以符合直觉的方式提供了以下功能:

\begin{itemize}
\item 线性运算:通过重载的运算符执行;支持并不完全。
\item 卷积:通过重载的 \lstinline{*}、\lstinline{*=} 运算符执行。
\item 求逆:通过 \lstinline{Poly Poly::inv() const} 执行。
\item 对数:通过 \lstinline{Poly log(const Poly &)} 执行。
\item 指数:通过 \lstinline{Poly exp(const Poly &)} 执行。
\end{itemize}

\subsubsection{注意}

在 VS Code 环境下使用 \lstinline{Poly} 的正确代码会产生异常报错。

\subsubsection{性能}

\begin{tabular}{|c|c|c|}
\hline
操作 & 规模 & 用时/标准时间 \\
\hline
卷积 & $|F| + |G| \le 2^{20}$ & 1.35 \\
\hline
求逆 & $n \le 10^5$ & 0.40 \\
\hline
对数 & $n \le 10^5$ & 0.55 \\
\hline
指数 & $n \le 10^5$ & 1.22 \\
\hline
\end{tabular}

\subsubsection{代码}

\lstinputlisting{src/poly.cpp}
