\documentclass[12pt]{ctexart}

\title{ICPC Templates}
\date{\today}
\author{Fei Pan}
\newcommand{\version}{v0.1}

\usepackage{amsfonts,amssymb}
\usepackage{graphicx}
\usepackage{amsmath}
\usepackage{listings}
\usepackage{xcolor}
\usepackage{geometry}
\usepackage{setspace}
\usepackage{fontspec}

\setmonofont{Hack}

\lstdefinestyle{cppstyle}{
    language=C++,
    backgroundcolor=\color{white},   % 选择背景色
    commentstyle=\color{green},      % 注释的颜色
    keywordstyle=\color{magenta},    % 关键词的颜色
    stringstyle=\color{red},         % 字符串的颜色
    basicstyle=\ttfamily\small,      % 基本字体
    breaklines=true,                  % 自动换行
    numbers=left,                     % 行号位置
    numberstyle=\tiny\color{gray},    % 行号的样式
    stepnumber=1,                    % 行号递增
    numbersep=5pt,                   % 行号和代码间距
    tabsize=4,                       % tab 宽度
    showspaces=false,                % 不显示空格
    showstringspaces=false,          % 不显示字符串中的空格
    showtabs=false,                  % 不显示制表符
    frame=single,
    emph={constexpr, nullptr, class, vector, pair, set, map, unordered_map},     % 自定义强调
    emphstyle=\color{magenta},         % 自定义强调样式
}
\lstset{style=cppstyle}

\geometry{a4paper}
\geometry{top=2cm} 
\geometry{bottom=1cm}
\geometry{left=1.5cm}
\geometry{right=1.5cm}

\setstretch{0.75}
\setcounter{tocdepth}{2}

\begin{document}
\maketitle
\begin{center}
    \textit{版本: }\version
\end{center}

\tableofcontents
\newpage

\section{说明}

定义执行一次以下代码花费的时间为一个标准时间。其在各个 OJ 上的具体值为:

\begin{itemize}
\item Codeforces: 202ms
\item AtCoder: 221ms
\item LibraryChecker: 176ms
\item Luogu: 242ms
\end{itemize}
\begin{lstlisting}
constexpr int N = 1e6;
std::set<int> s;
for (int i = 0; i < N; ++i) {
    s.insert(i);
}
\end{lstlisting}
\newpage

\section{计数}
\subsection{固定剩余系}

\subsubsection{代码}

\lstinputlisting{src/modint.cpp}

\subsection{多项式}

高效实现的多项式类,提供一切多项式支持。

\subsubsection{用法}

通过一个 \lstinline{ModInt} 的实例类来实例化 \lstinline{Poly}。对应的模数需为适合的 NTT 模数,例如:

\begin{itemize}
\item $167772161 = 5 \times 2^{25} + 1$
\item $469762049 = 7 \times 2^{26} + 1$
\item $998244353 = 119 \times 2^{23} + 1$
\item $1004535809 = 479 \times 2^{21} + 1$
\end{itemize}

该实现需要特别的原根,可以通过 \lstinline{primitive_root.cpp} 计算。对于 $998244353$ 可使用 $31$,对于 $1004535809$ 可使用 $702606812$。使用时,用适合的原根替换第 17 行的 \lstinline{31}。

\lstinline{Poly} 继承了 \lstinline{std::vector},因此支持一切 \lstinline{std::vector} 的功能。可能需要在 \lstinline{Poly} 的 public 区添加 \lstinline{using std::vector<Z>::method} 来导入相应方法。

该实现以符合直觉的方式提供了以下功能:

\begin{itemize}
\item 线性运算:通过重载的运算符执行;支持并不完全。
\item 卷积:通过重载的 \lstinline{*}、\lstinline{*=} 运算符执行。
\item 求逆:通过 \lstinline{Poly Poly::inv() const} 执行。
\item 对数:通过 \lstinline{Poly log(const Poly &)} 执行。
\item 指数:通过 \lstinline{Poly exp(const Poly &)} 执行。
\end{itemize}

\subsubsection{注意}

在 VS Code 环境下使用 \lstinline{Poly} 的正确代码会产生异常报错。

\subsubsection{性能}

\begin{tabular}{|c|c|c|}
\hline
操作 & 规模 & 用时/标准时间 \\
\hline
卷积 & $|F| + |G| \le 2^{20}$ & 1.35 \\
\hline
求逆 & $n \le 10^5$ & 0.40 \\
\hline
对数 & $n \le 10^5$ & 0.55 \\
\hline
指数 & $n \le 10^5$ & 1.22 \\
\hline
\end{tabular}

\newpage

\subsubsection{代码}

\lstinputlisting{src/poly.cpp}


\section{图论}
\subsection{最大流}

\subsubsection{代码}

\lstinputlisting{src/flow.cpp}

\subsection{二分图染色}

\subsubsection{代码}

\lstinputlisting{src/bipartite_coloring.cpp}


\section{数据结构}
\subsection{动态树}

\subsubsection{用法}

通过 \lstinline{LinkCutTree<Mono, TREE_N>} 实例化动态树类。

\begin{itemize}
\item \lstinline{Mono}:一个幺半群或半群类。需重载 \lstinline{*} 运算符。
\item \lstinline{TREE_N}:一个整数,值应\textbf{大于}需要使用的最大下标。
\end{itemize}

以下公共方法均允许不实现。

\begin{itemize}
\item \lstinline{link}:添加一条边。
\item \lstinline{cut}:切割一条边。
\item \lstinline{set}:设置一个节点的值。
\item \lstinline{query}:查询两点间路径上所有节点元素按顺序相乘的结果。
\item \lstinline{find}:查询一个节点所在树的根,可用于判断连通性。
\end{itemize}

\subsubsection{注意}

下标从 0 开始,但不支持使用 0 下标。

\subsubsection{性能}

$n, q = 2\times 10^5$ 约花费 1.49 个标准时间。

\subsubsection{代码}

\lstinputlisting{src/link_cut_tree.cpp}


\section{计算几何}
\subsection{最小圆覆盖}

\subsubsection{用法}

通过 \lstinline{SmallestCircle<T>} 实例化最小圆覆盖类,其中 \lstinline{T} 为一个浮点数类。通过 \lstinline{build} 指定一系列二维点,通过重载的函数调用运算符获取一个由圆心和半径组成的 \lstinline{std::pair}。修改成员 \lstinline{EPS} 以设置精度。

\subsubsection{注意}

提供给 \lstinline{build} 的二维点集应已被可靠地打乱顺序,以保证复杂度。

\subsubsection{性能}

时间复杂度为 $O(n)$。$n = 10^5$ 约花费 0.19 个标准时间。

\subsubsection{代码}

\lstinputlisting{src/smallest_circle.cpp}

\subsection{半平面交}

求解一系列半平面的交。

\subsubsection{用法}

半平面以 $Ax + By + C \ge 0$ 的形式表示。

\subsubsection{注意}

该实现的正确性依赖于外方框的存在。请确保传入的半平面中包含外方框。

\subsubsection{性能}

时间复杂度为 $O(n \log n)$。

\subsubsection{代码}

\lstinputlisting{src/half_plane_intersection.cpp}


\section{数论}
\subsection{原根}

\subsubsection{用法}

通过一个 \lstinline{ModInt} 的实例类实例化,返回一个对应的原根。该原根可用于 \lstinline{Poly}。

\subsubsection{代码}

\lstinputlisting{src/primitive_root.cpp}


\section{其他}
\subsection{幺半群区间积}

对于一个幺半群/半群构成的长度为 $n$ 的序列,离线计算 $q$ 个区间的积。该方法的优势在于效率,设单次乘法的复杂度为 $O(k)$,则该方法的复杂度为 $O(nk\log n + q)$ 且常数较小。

\subsubsection{用法}

需要一个幺半群/半群类 \lstinline{Mono}。在 \lstinline{monoid_product(vec, query, f)} 中,\lstinline{vec} 是一个 \lstinline{Mono} 的序列,\lstinline{query} 是一个区间的序列,\lstinline{f} 是一个对于每个长度大于等于 2 的询问区间调用的函数,对于每个询问区间 $q_i = (l, r)$,\lstinline{f} 均会以 \lstinline{f(L, R, i)} 的形式被调用,其中 \lstinline{L * R} 为 $q_i$ 的区间积。

\subsubsection{注意}

长度为 1 的询问区间会被忽略。下标从 0 开始。

\subsubsection{性能}

乘法大约仅会进行 $\dfrac{n \log n}{2}$ 次。

\subsubsection{代码}

\lstinputlisting{src/monoid_product.cpp}

\subsection{线性方程组求解}

求解一个线性方程组。

\subsubsection{用法}

向 \lstinline{solve_linear_equations(a, abs, is_zero)} 传入三个参数:
\begin{itemize}
\item \lstinline{a}:待解的方程组矩阵,其列数应比行数恰好大 1。
\item \lstinline{abs}:模长函数。
\item \lstinline{is_zero}:零等函数。
\end{itemize}

在 \lstinline{a} 上原地操作,将 \lstinline{a} 的系数部分转化为单位矩阵。如有自由元或无解,返回非零值。

\subsubsection{注意}

下标从 0 开始。

\subsubsection{性能}

\begin{tabular}{|c|c|c|}
\hline
\lstinline|T| & $n$ & 用时/标准时间 \\
\hline
\lstinline|mint| & $500$ & 0.69 \\
\hline
\lstinline|mint| & $1000$ & 4.09 \\
\hline
\lstinline|double| & $500$ & 0.23 \\
\hline
\lstinline|double| & $1000$ & 2.48 \\
\hline
\lstinline|long double| & $500$ & 0.69 \\
\hline
\lstinline|long double| & $1000$ & 7.58 \\
\hline
\end{tabular}

据信利用 \lstinline{std::bitset} 可以处理多至 $n = 5000$ 的异或线性方程组。

\subsubsection{代码}

\lstinputlisting{src/linear_equations.cpp}


\end{document}